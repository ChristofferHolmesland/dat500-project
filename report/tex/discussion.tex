\subsection{Data}
The data that can be downloaded from IMDb and OMDb is quite good. As already mentioned in section \ref{sec:method} having the box office value would have provided us another way to measure the impact of different actors. The fact that data changes every day resulting in missing data also raises the question of whether or not we can trust the data that we do get. If for example some people are listed as actors on a movie they have not acted in then the three calculate scores will not be correct. Especially the relation score could change a lot. We did not spend any time manually checking if any of these mistakes are present in the data, and during testing we did not spot and suspicious results. There are some additional actor features we wish were present in the data, for instance skin color, spoken languages or height. These could be used to filter the list of actors. If the data included regional information like "Hollywood", "Bollywood", etc. then we could have made a region-based system. The current results are dominated by actors based in Hollywood, which means that this tool cannot really be used by directors working in Bollywood.

\subsection{Limitations}
Our system typically recommends well known actors. They are often in great demand by directors meaning that they are often going to be busy or unavailable. Some of this information is available on the IMDb website as seen on the pages of Margot Robbie\cite{imdbMargot} and Tom Cruise\cite{imdbTom}. This is only available as a movie listings being either announced or in production, there is no way to see this in the datasets from IMDb. The result of this is that the director has to manually filter the list based on whether or not the actor is available. 