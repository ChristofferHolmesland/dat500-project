\subsection{Getting the data} 
First we had to get all of the required data. Most of it can be downloaded from the IMDB website(source to imdb/interfaces). This data does not include actor gender, movie plot or box office values. To determine the gender we used the list of known professions for every person. If they are known for being a actor they are assigned the label "0". Actresses are assigned the label "1". Some people in the dataset have never acted, for example directors or producers. They are removed from the data. There is no way to find the movie plot in the datasets, so we had to find an external source. One alternative is to use webscraping on the IMDB website. We tried doing this but quickly realized that it would not be possible for us because they stop responding to requests after too many in a given time period. We were able to get about 9~000 requests per hour. Since the dataset includes about 500~000 movies this would not be a good solution. It was also not possible for us to know if there were other rate limits e.g. 50~000 requests per day. Instead we used the OMDB API (source) to retrieve the movie plots. Using this API we were able to get all of the plots in 1 hour and 43 minutes. Our original idea was to also make a prediction on the box office value. After using the OMDB API to also find the box office values we saw that only 6381 movies have this value. We do not believe this is enough data to make proper predictions on (why?) so we did not use this data. Instead we decided to generate a prediction for the IMDB user score because this value is available for all of the movies in our dataset.

\subsection{Preprocessing}
The data from IMDB includes a lot of information we don't need and it also has missing values. It is also split over several files e.g.~title.basics.tsv and title.ratings.tsv. Every person in the dataset has an id, name, birthyear, deathyear, a list of primary professions and a list of titles they are known for working on. We replace the primary profession list with a number indicating the gender. We remove people who are dead or who has never been an actor or actress before. There are also some people who are actors but not known for any titles. They are also removed because we need that information to calculate a score. These steps reduce the number of people from 9.9 million to 200 thousand.

The data in the title.basics.tsv file is an id,	type, well known title,	original title,	adult, start year, end year, runtime, and a list of genres. The only relevant information for our project is the id, type and list of genres. The type is used to decide if we should keep it. We are only interested in looking at movies, represented by "movie" or "tvMovie" in the data. Shorts are typically news segments which would not be a good indicator of movie performance. If we included tv series then the actors in them would have inflated scores because there could be over 100 episodes for a given series. That would show that the person is good at that specific role, but not represent how well he or she can adapt to a new role. The runtime is not used in our system because it is missing for most of the titles. Some movies are missing the genre list and are thus removed because it is required for the system. This reduces the number of titles from 6.5 million to 580 thousand. This might seem like a lot, but most of it is because the data includes seperate entries for each season and episode for every tv series. In this step we also use the plot data from the OMDB API. Movies without the plot information are removed. This reduces the number further to 212 thousand (check this number again).

The movie ratings are in the title.ratings.tsv file from IMDB. The format is id, average rating and number of votes. Any title removed from title.basics.tsv is also removed from this data. The title.principals.tsv file contains a list of the most important people for each title. The fields are title id, ordering, actor id, category, job, characters. We use this information to determine what actors have worked together. We only look at the titles with id which was not removed in the previous step, and the ordering, category, job and characters fields are dropped because they provide no relevant information to us.

To improve the runtime performance of the search we calculate and store several actor dependent values during preprocessing. For every actor we calculate how many unique people they have worked with, and how many people they have worked with in total. This is used in our ranking function to give a measure of how well they work with other people. The other scores we calculate are genre scores. It is calculated as the average of the movie rating the actor has been in. The score is calculated for each genre. If actor A has acted in two movies, the first with score 5 with genres "romance" and "drama", the second with score 10 and genre "romance", then actor A has a "romance" score of 7.5 and a "drama" score of 5.

\subsection{Ranking algorithm}
The ranking algorithm is used to determine whether one actor is going to perform better than another. This is done by calculating a score for the actors, before combining them to get a score for the whole group. Our algorithm assumes the first actor description to be the primary actor and tries to maximise the cast rank based on this. The following is a description of our algorithm. The first step is to find every actor that matches the primary actor description. For every matching actor a score is calculated. It is a combination of the genre score, past acting relationship(maybe) and how similar the plots of the movies the actor has acted in are to the user plot. The genre score is taken as the average of the actor genre scores, given that they match the user genres. This is done because one actor might have a high score in war and action movies, but low in romance and drama. If the user is making a romance and drama movie, then the genre score is taken as the average of just those two scores. (write about past acting relationship if we decide to use it). To find how similar the plots are the python library gensim is used to compare. The plot of every movie the actor has been in is compared to the user plot. The maximum similarity score is used for the actor score. The similarity from gensim is the cosine similarity in the range (-1, 1). To make the average of the genre score and the similarity equally important for the actor score the cosine similarity bound is changed to (0, 10). Finally the actor score is changed to be in the interval (0, 1). Thus the expression for the score is: $actor(id)=\frac{1}{20}(avg(genre_score) + 5\cdot(max(similarity)+1))$.

To find the other actors the procedure above is repeated on the actors matching the other descriptions. Because we consider it beneficial to cast actors who have worked together before, the actor scores of the secondary actors are increased by X if they have worked with the primary actor previously. To calculate the cast score we also make a prediction of the IMDB rating that the resulting movie will have if the selected actors are in it. The prediction is calculated from the average of the actors average genre scores. The expression for the cast score is: $cast(ids)=\frac{1}{\#ids}\sum_{i=1}^{\#ids}worked(ids_i, ids_1)\cdot actor(ids_i) + rating(ids)$. The $worked$ function returns X if they have worked together before, 0 otherwise. The score of the primary actor is not increased by X.

The cast score is used to rank the groups. A higher score means a better groups. The groups are sorted in descending order and returned to the user.