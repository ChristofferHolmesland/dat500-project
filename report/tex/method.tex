\subsection{Design} 
There are three parts to the system. First we had to get all of the required data. Most of it can be downloaded from the IMDB website(source to imdb/interfaces). This data does not include actor gender, movie plot or box office values. To determine the gender we used the list of known professions for every person. If they are known for being a actor they are assigned the label "0". Actresses are assigned the label "1". Some people in the dataset have never acted, for example directors or producers. They are removed from the data. There is no way to find the movie plot in the datasets, so we had to find an external source. One alternative is to use webscraping on the IMDB website. We tried doing this but quickly realized that it would not be possible for us because they stop responding to requests after too many in a given time period. We were able to get about 9~000 requests per hour. Since the dataset includes about 500~000 movies this would not be a good solution. It was also not possible for us to know if there were other rate limits e.g. 50~000 requests per day. Instead we used the OMDB API (source) to retrieve the movie plots. Using this API we were able to get all of the plots in 1 hour and 43 minutes. Our original idea was to also make a prediction on the box office value. After using the OMDB API to also find the box office values we saw that only 6381 movies have this value. We do not believe this is enough data to make proper predictions on (why?) so we did not use this data. Instead we decided to generate a prediction for the IMDB user score because this value is available for all of the movies in our dataset.

Concisely present your design. Focus on novel aspects,
but avoid implementation details. Use pseudo-code and figures to better
explain your design, but also present and explain these in the text.
%
To assist in evaluating your design choices, it may be relevant to describe
several distinct \textit{design alternatives} that can later be compared.

\subsection{Analysis}

\subsection{Optimization} Explain how you optimized your design and 
adjusted it to specific situations.
%
Generally, as important as the final results is to show
that you took a structured, organized approach to the optimization
and that you explain why you did what you did.
%
Be careful to argue, why your optimization does not break the 
correctness of your design, established above.
%
It is often a good strategy to explain a design or protocol in stepwise refinements,
so as to more easily convince the reader of its correctness. 

\subsection{Implementation} It is not necessary to "explain" your code. 
However, in some cases it may be relevant to highlight 
additional contributions given by your implementation.
Examples for such contributions are:
\begin{itemize}
\item \emph{Abstractions and modules}: If your implementation is nicely separated into interacting
modules with separated responsibilities you could explain this structure,
and why it is good/better than some other alternative structure.
\item \emph{Optimization}: If you spend significant time optimizing your code, e.g.~using profiling tools,
the result of such optimization may be presented as a contribution. In this case, reason, 
why the optimized code works better.
\item \emph{Evaluation framework}: If you implemented a framework or application to evaluate your implementation against existing work, or in a specific scenario, this framework may be presented as a contribution.
\end{itemize}

Make sure to cite all external resources used.